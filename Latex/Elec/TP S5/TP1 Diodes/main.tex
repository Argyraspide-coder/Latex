\documentclass{article}
\usepackage{graphicx}
\usepackage{amsmath,amssymb}
\usepackage[french]{babel}
\usepackage{graphicx}


\usepackage[utf8]{inputenc}
\usepackage{siunitx}
\usepackage[european, straightvoltages, RPvoltages]{circuitikz}
\usetikzlibrary{babel}

\counterwithin{figure}{section}
\counterwithin{figure}{subsection}
\usepackage[hidelinks]{hyperref}

% chktex-file 1
% chktex-file 26
% chktex-file 8
% chktex-file 35
% chktex-file 24

\title{- Rapport de Travaux Pratiques -\\
Élec. Analogique : \\
TP1 - Diode }
\author{Gabriel Châtelain - Louison Bost}
\date{Novembre-Décembre 2024}

\begin{document}

\maketitle
\tableofcontents

\bigskip

\textbf{Toutes les références de la table sont liées vers leurs parties correspondantes}

\newpage

\section{Introduction}

Dans ce TP, on se propose d'étudier plusieurs circuits mettant en jeux plusieurs types de diode. Le but sera de mesurer leurs caractéristiques, puis de caractériser les circuits correspondants à l'aide de mesures et d'analyses théoriques.

\section{Diodes de Signal}

\subsection{Diode de signal rapide 1N4148}

On étudie le circuit suivant :
\begin{center}
\begin{circuitikz}
  \draw (0,0) to[vsource,v^=VIN $E$,i=$i$] (0,3)
        to[R, l^= {$ R_1 = 470~\Omega$},v_={$U_R$} ] (3,3)
        to[D,l=D1, l=\footnotesize 1N4148, v_ = {$U_D$}] (3,0)
        -- (0,0);
  \draw (1.5,0) node[ground]{};
    
\end{circuitikz}
\end{center}

\subsubsection{Caractéristique}

On établi la caractéristique suivante : 

\begin{figure}[h]
   \centering
    \includegraphics[width=0.5\linewidth]{placeholder.jpg}
    \caption{Caractéristique de la Diode}
    \label{fig : Caractéristique diode}
\end{figure}



On relève alors les valeurs :
\[V_0 = \text{ et } R_f=\]

\subsubsection{Comparaison avec les valeurs Datasheet}

On trouve sur internet que \[V_{0,ref} = \text{ et } R_{f,ref}=\]

En considérant les incertitudes de mesures, on calcule alors le Z-score :

\[Z=\frac{|x_{exp}-x_{ref}|}{u}\]

\[Z_{V_0}~=~A~;~Z_{R_f}~=~A~\]

La mesure est donc validée ($Z$ $\leq$ 2).

\newpage

\subsection{Redressement Mono-Alternance}

On étudie le circuit suivant :
\begin{center}
\begin{circuitikz}
  \draw (0,0) to[sI,v^={$u_4$},i={$i$}] (0,3)
        to[D,l=D1, l=\footnotesize 1N4148, v_ = {$u_D$}] (5,3)
        to[R, l^= {$ R_1 = 470~\Omega$},v_={$u_R$} ] (5,0)
        -- (0,0);
  \draw (2.5,0) node[ground]{};
  \node[align=right] at (-2,0.5) {sin.\\ $V_A = 10V$\\ $f=1.0~\text{kHz}$};    
\end{circuitikz}
\end{center}

\subsubsection{Comportement sur une période}

On observe le comportement suivant sur une période :

\begin{figure}[h]
   \centering
     \includegraphics[width=0.5\linewidth]{placeholder.jpg}
     \caption{Tension aux bornes de $R_1$ sur une période}
     \label{fig : Ref Name}
\end{figure}

On remarque alors que de $t_0$ à $t_1$, la tension appliquée $V_4$ est suffisante pour que le courant passe dans la diode.
Autrement, la diode est bloquée, aucun courant ne passe.

\subsubsection{Chargement d'une capacité}

On modifie le circuit de la façon suivante :

\begin{center}
    \begin{circuitikz}
        \draw (0,0) to[sI,v^={$v_4$},i={$i$}] (0,3)
        to[D,l=D1, l=\footnotesize 1N4148, v_ = {$u_D$}] (5,3)
        to[R, l^= {$ R_1 = 470~\Omega$},v_={$u_R$} ] (5,0)
        -- (0,0);
        \draw (5,3) to (8.5,3)
        to[C,l={C= $1.0\mu F$},v_={$u_C$}] (8.5,0)
        -- (0,0);

  \draw (2.5,0) node[ground]{};
  \node[align=right] at (-2,0.5) {sin.\\ $V_A = 10V$\\ $f=1.0~\text{kHz}$};     
    \end{circuitikz}
\end{center}

On relève alors les tensions suivantes sur une période :


On peut alors remarquer que le comportement du circuit se décline en 3 étapes répétées :

\begin{itemize}
    \item Situation initiale : $v_4 = 0$ la diode est bloquée, C est en décharge dans $R_1$
    \item $v_4$ augmente et $v_4 \geq V_0$, la tension de seuil de D ; la diode est passante,
                le condensateur diminue la vitesse de sa décharge puis entre en charge.
    \item $v_4$ diminue puis $v_4 \leq V_0$, la diode est bloquée, C se décharge dans $R_1$ ; on est ramené à la situation initiale
\end{itemize}

\newpage

\subsection{Redressement bi-alternance}

On travaille à présent sur le circuit suivant :

\begin{center}
    \begin{circuitikz}
        \draw (0,0) to[sI,v^={$v_3$},i={$i$}] (0,3)
        to (3,3)
        to (3,2.5);
        \draw (0,0)
        to (3,0)
        to (3,0.5);

        \draw (2,1.5) to[D,l^={$D_1$}] (3,2.5);
        \draw (3,2.5) to[D,l^={$D_2$}] (4,1.5);
        \draw (2,1.5) to[D,l_={$D_3$}] (3,0.5);
        \draw (3,0.5) to[D,l_={$D_4$}] (4,1.5);

        \draw (2,1.5)
        to (1,1.5)
        to (1,-1)
        to (8,-1) ;

        \draw (8,3) to[R,v={$u_{R_3}$},invert,l={R=1 k$\Omega$}] (8,-1);
        
        \draw (8,3)
        to (6,3)
        to (6,1.5)
        to (4,1.5);

          \draw (0.5,0) node[ground]{};
    \end{circuitikz}
\end{center}

On précise que toutes les diodes ont le même modèle que celle exploitée précedement dans le TP.

On relève alors les observations suivantes :

\begin{figure}[h]
     \centering
         \includegraphics[width=0.5\linewidth]{placeholder.jpg}
         \caption{$u_{R_3}$ sur une période}
         \label{fig : Ref Name}
\end{figure}

\newpage

On distingue 4 temps sur une période complète :

\bigskip

\begin{itemize}
  \item Situation initiale : $v_3$ = 0, aucun courant ne circule, toutes les diodes sont bloquées
  \item $v_3$ augmente, puis $v_3$ $\geq$ 2$V_0$, alors les diodes $D_2$ et $D_3$ sont passantes, un courant circule dans R !
  \item $v_3$ atteint un maximum puis diminue jusqu'à $v_3$ $\leq$ 2$V_0$, les diodes redeviennent toutes bloquantes, aucun courant ne circule.
  \item $v_3$ diminue puis $v_3$ $\leq$ -2$V_0$, alors les diodes $D_1$ et $D_4$ sont passantes, un courant circule à nouveau dans R !
  \item $v_3$ atteint un minimum puis augmente jusqu'à $v_3$ $\geq$ -2$V_0$, les diodes redeviennent toutes bloquantes, aucun courant ne circule.
              On est reconduis à la situation initiale.
\end{itemize}

\subsubsection{Charge d'un condensateur en Redressement bi-alternance}

On travaille à présent sur le circuit suivant :

\begin{center}
    \begin{circuitikz}
        \draw (0,0) to[sI,v^={$v_3$},i={$i$}] (0,3)
        to (3,3)
        to (3,2.5);
        \draw (0,0)
        to (3,0)
        to (3,0.5);

        \draw (2,1.5) to[D,l^={$D_1$}] (3,2.5);
        \draw (3,2.5) to[D,l^={$D_2$}] (4,1.5);
        \draw (2,1.5) to[D,l_={$D_3$}] (3,0.5);
        \draw (3,0.5) to[D,l_={$D_4$}] (4,1.5);

        \draw (2,1.5) to (1,1.5)
        to (1,-1)
        to (8,-1) ;

        \draw (8,3) to[R,l={R=1 k$\Omega$},v={$u_{R_3}$},invert] (8,-1) ;

        \draw (8,3)
        to (6,3)
        to (6,1.5)
        to (4,1.5);

        \draw (0.5,0) node[ground]{};
        \draw (8,-1) to (12,-1);

        \draw (12,3) to[C,l={C= $1.0\mu F$},v_={$u_C$}] (12,-1) ;
        \draw (12,3) to (8,3);
    \end{circuitikz}
\end{center}

\begin{figure}[h]
   \centering
     \includegraphics[width=0.5\linewidth]{placeholder.jpg}
     \caption{Tension $u_C$ sur une période de $v_3$}
     \label{fig : Redressement bi-alternance capacité}
\end{figure}

\bigskip

\textbf{Analyse}

On distingue 4 temps sur une période complète :

\bigskip

\begin{itemize}
  \item Situation initiale : $v_3$ = 0, aucun courant ne circule, toutes les diodes sont bloquées, $C$ se décharge dans $R$.
  \begin{center}
    \begin{circuitikz}[scale=0.70]
        \draw (0,0) to[sI,v^={$v_3$},i={$i$}] (0,3)
        to (3,3)
        to (3,2.5);
        \draw (0,0)
        to (3,0)
        to (3,0.5);

        \draw (2,1.5) to (1,1.5)
        to (1,-1)
        to (8,-1) ;

        \draw (8,3) to[R,l={R=1 k$\Omega$},v={$u_{R_3}$},invert] (8,-1) ;

        \draw (8,3)
        to (6,3)
        to (6,1.5)
        to (4,1.5);

        \draw (0.5,0) node[ground]{};
        \draw (8,-1) to (12,-1);

        \draw (12,3) to[C,l={C= $1.0\mu F$},v_={$u_C$}] (12,-1) ;
        \draw (12,3) to (8,3);
    \end{circuitikz}
\end{center}

\bigskip


  \item $v_3$ augmente, puis $v_3$ $\geq$ 2$V_0$, alors les diodes $D_2$ et $D_3$ sont passantes, $C$ se décharge de moins en moins puis se charge.
  \begin{center}
    \begin{circuitikz}[scale=0.70]
        \draw (0,0) to[sI,v^={$v_3$},i={$i$}] (0,3)
        to (3,3)
        to (3,2.5);
        \draw (0,0)
        to (3,0)
        to (3,0.5);

        \draw (3,2.5) to[D,l^={$D_2$}] (4,1.5);
        \draw (2,1.5) to[D,l_={$D_3$}] (3,0.5);


        \draw (2,1.5) to (1,1.5)
        to (1,-1)
        to (8,-1) ;

        \draw (8,3) to[R,l={R=1 k$\Omega$},v={$u_{R_3}$},invert] (8,-1) ;

        \draw (8,3)
        to (6,3)
        to (6,1.5)
        to (4,1.5);

        \draw (0.5,0) node[ground]{};
        \draw (8,-1) to (12,-1);

        \draw (12,3) to[C,l={C= $1.0\mu F$},v_={$u_C$}] (12,-1) ;
        \draw (12,3) to (8,3);
    \end{circuitikz}
\end{center}

\bigskip

  \item $v_3$ atteint un maximum puis diminue jusqu'à $v_3$ $\leq$ 2$V_0$, les diodes redeviennent toutes bloquantes, $C$ se décharge dans $R$.
  \begin{center}
    \begin{circuitikz}[scale=0.70]
        \draw (0,0) to[sI,v^={$v_3$},i={$i$}] (0,3)
        to (3,3)
        to (3,2.5);
        \draw (0,0)
        to (3,0)
        to (3,0.5);



        \draw (2,1.5) to (1,1.5)
        to (1,-1)
        to (8,-1) ;

        \draw (8,3) to[R,l={R=1 k$\Omega$},v={$u_{R_3}$},invert] (8,-1) ;

        \draw (8,3)
        to (6,3)
        to (6,1.5)
        to (4,1.5);

        \draw (0.5,0) node[ground]{};
        \draw (8,-1) to (12,-1);

        \draw (12,3) to[C,l={C= $1.0\mu F$},v_={$u_C$}] (12,-1) ;
        \draw (12,3) to (8,3);
    \end{circuitikz}
\end{center}

\bigskip
  
  \item $v_3$ diminue puis $v_3$ $\leq$ -2$V_0$, alors les diodes $D_1$ et $D_4$ sont passantes, $C$ se décharge de moins en moins puis se charge, on remarque que la polarité constante en sortie du redresseur garantie que l'on charge toujours $C$ avec la même polarité.
  \begin{center}
    \begin{circuitikz}[scale=0.70]
        \draw (0,0) to[sI,v^={$v_3$},i={$i$}] (0,3)
        to (3,3)
        to (3,2.5);
        \draw (0,0)
        to (3,0)
        to (3,0.5);

        \draw (2,1.5) to[D,l^={$D_1$}] (3,2.5);
        \draw (3,0.5) to[D,l_={$D_4$}] (4,1.5);

        \draw (2,1.5) to (1,1.5)
        to (1,-1)
        to (8,-1) ;

        \draw (8,3) to[R,l={R=1 k$\Omega$},v={$u_{R_3}$},invert] (8,-1) ;

        \draw (8,3)
        to (6,3)
        to (6,1.5)
        to (4,1.5);

        \draw (0.5,0) node[ground]{};
        \draw (8,-1) to (12,-1);

        \draw (12,3) to[C,l={C= $1.0\mu F$},v_={$u_C$}] (12,-1) ;
        \draw (12,3) to (8,3);
    \end{circuitikz}
\end{center}

\bigskip
  
  \item $v_3$ atteint un minimum puis augmente jusqu'à $v_3$ $\geq$ -2$V_0$, les diodes redeviennent toutes bloquantes, $C$ se décharge dans $R$.
              On est reconduis à la situation initiale.

\end{itemize}

\newpage

\section{Diodes Zener}

On travaille à présent avec les diodes Zener BZX C7V5 ou BZX C6V8

\subsection{Simple Ecretage}

On étudie le circuit suivant :
\begin{center}
\begin{circuitikz}
  \draw (0,0) to[sI,v^={$u_4$},i={$i$}] (0,3)
        to[R, l^= {$ R_1 = 470~\Omega$},v_={$u_R$} ] (5,3)
        to[zDo,l=D1, l={$D$}, v_ = {$u_D$},invert] (5,0)
        -- (0,0);
  \draw (2.5,0) node[ground]{};
  \node[align=right] at (-2,0.5) {sin.\\ $V_A = 10V$\\ $f=1.0~\text{kHz}$};    
\end{circuitikz}
\end{center}

On relève les mesures suivantes sur une période :


\begin{figure}[h]
   \centering
     \includegraphics[width=0.5\linewidth]{placeholder.jpg}
     \caption{Tensions $u_4$ et $u_D$ sur une période}
     \label{fig : Ref Name}
\end{figure}

\textbf{Analyse}

On observe que : 

\subsection{Double Ecretage}

Intuitivement, il suffirait de brancher une seconde diode dans le sens opposé et parallèlement à la première :
\begin{center}
\begin{circuitikz}
  \draw (0,0) to[sI,v^={$u_4$},i={$i$}] (0,3)
        to[R, l^= {$ R_1 = 470~\Omega$},v_={$u_R$} ] (5,3)
        to[zDo,l=D1, l={$D$}, v_ = {$u_D$},invert] (5,0)
        -- (0,0);
  \draw (5,3)
        to (7,3)
        to[zDo,l=D1, l={$D$}, v_ = {$u_D$}] (7,0)
        to (5,0) ;
  \draw (2.5,0) node[ground]{};
  \node[align=right] at (-2,0.5) {sin.\\ $V_A = 10V$\\ $f=1.0~\text{kHz}$};    
\end{circuitikz}
\end{center}

On construit le circuit et on observe les mesures suivantes :
  
\begin{figure}[h]
   \centering
     \includegraphics[width=0.5\linewidth]{placeholder.jpg}
     \caption{Mesures de vérification du double ecretage}
     \label{fig : double ecretage exp}
\end{figure}

\section{Diode électro-luminescente}

On étudie le circuit suivant :
\begin{center}
\begin{circuitikz}
  \draw (0,0) to[vsource,v^=VE $E$,i=$i$] (0,3)
        to[R, l^= {$ R_5 = 1k~\Omega$},v_={$U_R$} ] (5,3)
        to[leDo,l=D1, l={$D_6$}, v_ = {$U_{D_6}$}] (5,0)
        -- (0,0);
  \draw (1.5,0) node[ground]{};
    
\end{circuitikz}
\end{center}

\textbf{a) Allumage}

Experimentalement, la DEL s'allume dès que $E ~>~A~V$, soit $U_{D_6} ~>~A~V$.
On trouve sur internet :
\bigskip


\textbf{b) Allumage avec moins de résistance}

Experimentalement, la DEL s'allume dès que $E ~>~A~V$, soit $U_{D_6} ~>~A~V$.


\end{document}
