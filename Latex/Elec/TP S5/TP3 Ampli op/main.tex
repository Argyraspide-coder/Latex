\documentclass{article}
\usepackage{graphicx}
\usepackage{amsmath,amssymb}
\usepackage[french]{babel}
\usepackage{graphicx}

\usepackage{enumitem}
\usepackage[utf8]{inputenc}
\usepackage{siunitx}
\usepackage[european, straightvoltages, RPvoltages]{circuitikz}
\usetikzlibrary{babel}

\counterwithin{figure}{section}
\counterwithin{figure}{subsection}
\usepackage[hidelinks]{hyperref}

% chktex-file 1
% chktex-file 26
% chktex-file 8
% chktex-file 35
% chktex-file 24
% chktex-file 36

\title{- Rapport de Travaux Pratiques -\\
Élec. Analogique : \\
TP3 - Ampli. Op. }
\author{Gabriel Barbe - Louison Bost}
\date{07/11/2025}

\begin{document}

\maketitle
\tableofcontents

\bigskip

\textbf{Toutes les références de la table sont liées vers leurs parties correspondantes}

\newpage

\section{Travaux préliminaires}

Dans ce TP, on se propose d'étudier un amplificateur opérationnel classique utilisé dans plusieurs montages classiques.

\subsection{ Caractéristiques techniques de l'AOP}

On relève les caractéristiques suivantes à partir de la Datasheet Microchip :

\begin{itemize}[label=\large - \normalsize]
  \item \textbf{Bande passante} pour une amplification unitaire : 0 - 10 MHz
  \item \textbf{Décalage type de tension} à l'entrée, et ses variations selon la température :
  \begin{figure}[h]
     \centering
       \includegraphics[width=\linewidth]{Capture d'écran 2025-11-07 011958.png}
       \caption{Décalage en entrée pour différentes températures}
       \label{graphe Datasheet tension entree}
  \end{figure}

       Sur ce point précis, on observe bien un offset réél, mais de faible valeur, inférieur à 0.5 mV en valeur absolue et dans des conditions extrêmes (125°c).
       En condition "raisonnables", il est inférieur à 0.15mV. En pratique, cela signifie qu'il faut nous assurer que les amplitudes des signaux à l'entrée seront grands devant ces valeurs.

  \item \textbf{Courant de polarisation} en conditions standard à 25°c : 1 mA
  \item \textbf{Courant de décalage} en entrée : $\pm$ 1 pA
  \item \textbf{Impédance} en entrée : $ \sim ~ 10^{13}~\Omega  $ 
  \item \textbf{Gain} en boucle ouverte : $\sim $ 110 dB 
  \item \textbf{Le taux de rejection} du mode commun : 90 dB.
  
  Comme le nom de ce paramètre n'engendre pas une compréhension immédiate,
  on se permet de rappeler que cette valeur décrit la capacité 
  du système à rejeter des signaux présents aux  deux entrée (d'où le commun), par exemple des interférences électro-magnétiques.
  \item \textbf{Plage d'alimentation} : 2.5V - 5.5V
  \item \textbf{Tensions de Saturation} pour $V_{DD}~=~5V$ : 15 m$V$ (min), 5.3 $V$ (max).
  \item \textbf{Slew rate} : 7.0 $V$/$\mu s$, c'est la pente maximale du signal en sortie.
      \end{itemize}


\section{Montage amplificateur non inverseur}

On considère le montage suivant :

\begin{center}
\begin{circuitikz}

\draw
  (0, 0) node[op amp] (opamp) {}
  (opamp.-) to ++(-1,0)
  ++(0,2)
  ++(-4,0)
  to[vsource,v_<={$V_2$},] ++(0,-2) 
  node[ground]{} ;
\draw
  (opamp.-)
  ++(-1,0)
  ++(0,2)
  ++(-4,0)
  to[R,l={$R_1 = 10$ k$\Omega$},i={$i_{R_1}$},v={$v_{R_1}$}] ++(4,0) ;

\draw
  (opamp.-) to ++(-1,0)
  to ++(0,2)
  to[R,label={$R_2 = 100$ k$\Omega$},i={$i_{R_2}$},v={$v_{R_2}$}] ++(4,0)
  to ++(0,-2.5)
  to (opamp.out) ;
\draw 
  (opamp.+) to ++(-1,0)
  to[sI,v^<={$v_E$}] ++(0,-1.5)
  to[vsource,v^<={$V_1$}] ++(0,-1.5)
  node[ground]{} ;
\draw (opamp.up) 
  to ++(0,0.5)
  node[tground]{}
  ++(0,0.2) node[]{5V};
\draw (opamp.down) 
  to ++(0,-0.5)
  node[ground]{};
\draw (opamp.out) 
  to ++(2,0)
  ++(0.5,0) node[]{$v_{out}$};


\end{circuitikz}
\end{center}

Sauf mention contraire, on considère les valeurs suivantes pour le circuit :
\begin{itemize}
  \item $R_1$ = 10 k$\Omega$
  \item $R_2$ = 100 k$\Omega$
  \item $V_1$ = 2.5 $V$
  \item $V_2$ = 2.5 $V$
  \item $v_E$ = $A \cos (2\pi f t)$ , $f$ = 1000 Hz, $A=v_{e,cc}$  = 100 m$V$ 
\end{itemize}

\subsection{Premier test d'amplification}

On effectue une première amplification, avec les paramètres standards mentionnés. On obtient le résultat suivant :

\begin{figure}[h]
   \centering
     \includegraphics[width=0.5\linewidth]{placeholder.jpg}
     \caption{Amplification signal standard}
     \label{Amplification signal standard}
\end{figure}

\newpage

\textbf{Commentaire : }

On observe ici un gain de X dB, ce qui correspond à la valeur attendue de Y dB (voir \hyperref[sec:annexe]{\underline{étude théorique annexe}})

\subsection{Variation de $V_1$}

\label{sec:retour}

On va fixer la valeur de $V_2$ à 2.5 $V$, et faire varier $V_1$ progressivement de 0 à 5$V$.

On observe alors que :

On en conclue que :


\subsection{Variation de $V_2$}

On va fixer la valeur de $V_1$ à 2.5 $V$, et faire varier $V_2$ progressivement de 0 à 5$V$.

On observe alors que :

On en conclue que :

\subsection{Variation de $v_e$}

On va fixer la valeur de $V_1$ et $V_2$ à 2.5 $V$, et faire varier $A$ progressivement de 0 à 5$V$. ($A=v_{e,cc}$  est l'amplitude)

On observe alors que :

On en conclue que :

\subsection{Étude de l'amplification en fonction de l'amplitude en entrée}

\begin{figure}[h]
   \centering
     \includegraphics[width=0.5\linewidth]{placeholder.jpg}
     \caption{Amplitude en sortie en fonction de l'amplitude en entrée}
     \label{vs en fonction de ve}
\end{figure}

\textbf{Commentaire :}

\section{Réponse en fréquence}

\subsection{Diagramme de Bode, $R_2$ = 100 k$\Omega$}

On trace le diagramme de Bode pour la bande 100 Hz - 5 MHz. On porte $A=v_{e,cc}$ à 300 m$V$, tous les autres 
paramètres du montage sont conservés.

\begin{figure}[h]
   \centering
     \includegraphics[width=0.5\linewidth]{placeholder.jpg}
     \includegraphics[width=0.5\linewidth]{placeholder.jpg}
     \caption{Diagramme de Bode pour $R_2$ = 100 k$\Omega$}
     \label{Bode 100k}
\end{figure}

\newpage

\subsection{Diagramme de Bode, $R_2$ = 330 k$\Omega$}

On trace  à nouveau le diagramme de Bode pour la bande 100 Hz - 5 MHz. On porte la valeur de $R_2$ à 330 k$\Omega$
 et $A=v_{e,cc}$ à 100 m$V$ tous les autres paramètres du montage sont conservés.

\begin{figure}[h]
   \centering
     \includegraphics[width=0.5\linewidth]{placeholder.jpg}
     \includegraphics[width=0.5\linewidth]{placeholder.jpg}
     \caption{Diagramme de Bode pour $R_2$ = 330 k$\Omega$}
     \label{Bode 330k}
\end{figure}

\subsection{Interpretations sur le produit Gain-Bande}

\newpage

\section{Polarisation automatique}

On considère maintenant le montage suivant :

\begin{center}
\begin{circuitikz}

\draw
  (0, 0) node[op amp] (opamp) {}
  (opamp.-) to ++(-1,0)
  to ++(0,2)
  to[R,l={$R_1~=~1$k$\Omega$}] ++(-3,0)
  to[C,l={$C_1~=~1\mu F$}] ++(-3,0)
  node[]{}
  ++(-0.3,0)
  node[]{$v_e$} ;

\draw
  (0, 0) node[op amp] (opamp) {}
  (opamp.-) to ++(-1,0)
  to ++(0,2)
  to[R,label={$R_2 = 10$ k$\Omega$}] ++(4,0)
  to ++(0,-2.5)
  to (opamp.out) ;

\draw
  (0, 0) node[op amp] (opamp) {}
  (opamp.-) to ++(-1,0)
  to ++(0,4)
  to[C,label={$C_2 = 1 \eta F$}] ++(4,0)
  to ++(0,-4) ;

\draw 
  (opamp.+) to ++(-1,0)
  node[]{} 
  ++(-1,0)
  node[]{$V_1 = 2.5V$} ;

\draw (opamp.up) 
  to ++(0,0.5)
  node[tground]{}
  ++(0,0.2) node[]{5V};

\draw (opamp.down) 
  to ++(0,-0.5)
  node[ground]{};

\draw (opamp.out) 
  to ++(2,0)
  ++(0.5,0) node[]{$v_{s}$};


\end{circuitikz}
\end{center}

\subsection{Sortie de l'entrée nulle}

On ne met aucune tension en entrée, soit $v_e$ = 0$V$.

On obtient les résultats suivants :

\begin{itemize}
  \item $V_+$ =
  \item $V_-$ =
  \item $V_s$ =
\end{itemize}

\textbf{Commentaire :}


\newpage

\subsection{Diagramme de Bode}

On trace  à nouveau le diagramme de Bode pour la bande 100 Hz - 5 MHz.

\begin{figure}[h]
   \centering
     \includegraphics[width=0.5\linewidth]{placeholder.jpg}
     \includegraphics[width=0.5\linewidth]{placeholder.jpg}
     \caption{Diagramme de Bode}
     \label{Bode 330k}
\end{figure}

\newpage

\section{Annexe : Étude théorique}
\label{sec:annexe}

\subsection{Montage amplificateur non-inverseur}

On considère le montage suivant :

\begin{center}
\begin{circuitikz}

\draw
  (0, 0) node[op amp] (opamp) {}
  (opamp.-) to ++(-1,0)
  ++(0,2)
  ++(-4,0)
  to[vsource,v_<={$V_2$},] ++(0,-2) 
  node[ground]{} ;
\draw
  (opamp.-)
  ++(-1,0)
  ++(0,2)
  ++(-4,0)
  to[R,l={$R_1 = 10$ k$\Omega$},i={$i_{R_1}$},v={$v_{R_1}$}] ++(4,0) ;

\draw
  (opamp.-) to ++(-1,0)
  to ++(0,2)
  to[R,label={$R_2 = 100$ k$\Omega$},i={$i_{R_2}$},v={$v_{R_2}$}] ++(4,0)
  to ++(0,-2.5)
  to (opamp.out) ;
\draw 
  (opamp.+) to ++(-1,0)
  to[sI,v^<={$v_E$}] ++(0,-1.5)
  to[vsource,v^<={$V_1$}] ++(0,-1.5)
  node[ground]{} ;
\draw (opamp.up) 
  to ++(0,0.5)
  node[tground]{}
  ++(0,0.2) node[]{5V};
\draw (opamp.down) 
  to ++(0,-0.5)
  node[ground]{};
\draw (opamp.out) 
  to ++(2,0)
  ++(0.5,0) node[]{$v_{out}$};


\end{circuitikz}
\end{center}

On cherche, pour les besoins de l'étude, à déterminer le gain en tension des composantes sinusoïdales.
On notera $v_+,v_-$ les tensions, respectivement, aux entrées + et - de l'AOP.

On suppose que $v_+~=~v_-$, car le montage est en rétroaction négative. 

De plus, on considère que $i_-~=~0$ car l'impédance en entrée de l'AOP est très grande devant les autres
 impédance mises en jeu dans le montage

Il vient alors : 
\[i_{R_2} = i_{R_1} \]

On a ainsi un diviseur de tension ; 

\[v_- = \frac{R_1}{R_1 + R_2} \bigg(v_{out}-V_2\bigg)\]

soit :

\[v_E + V_1 = \frac{1}{11} \bigg(v_{out}-V_2\bigg)\]

Ainsi, on obtient :

\[v_{out} = 11V_1 + V_2 + 11v_E \]

D'où :

\[G_{\text{dB}} = 20\log(11) = 20.8 \text{ dB}\]

Pour le cas $R_2$ = 330 k$\Omega$, on obtient :

\[G_{\text{dB}} = 20\log(34) = 30.6 \text{ dB}\]

Par ailleurs, on peut voir que l'on s'attend à observer des variations linéaires importantes des composantes continues en sorties
si l'on modifie $V_1$ ou $V_2$

\bigskip

\hyperref[sec:retour]{\underline{Retour à l'étude pratique}}

\end{document}
